% Options for packages loaded elsewhere
\PassOptionsToPackage{unicode}{hyperref}
\PassOptionsToPackage{hyphens}{url}
%
\documentclass[
]{article}
\usepackage{lmodern}
\usepackage{amssymb,amsmath}
\usepackage{ifxetex,ifluatex}
\ifnum 0\ifxetex 1\fi\ifluatex 1\fi=0 % if pdftex
  \usepackage[T1]{fontenc}
  \usepackage[utf8]{inputenc}
  \usepackage{textcomp} % provide euro and other symbols
\else % if luatex or xetex
  \usepackage{unicode-math}
  \defaultfontfeatures{Scale=MatchLowercase}
  \defaultfontfeatures[\rmfamily]{Ligatures=TeX,Scale=1}
\fi
% Use upquote if available, for straight quotes in verbatim environments
\IfFileExists{upquote.sty}{\usepackage{upquote}}{}
\IfFileExists{microtype.sty}{% use microtype if available
  \usepackage[]{microtype}
  \UseMicrotypeSet[protrusion]{basicmath} % disable protrusion for tt fonts
}{}
\makeatletter
\@ifundefined{KOMAClassName}{% if non-KOMA class
  \IfFileExists{parskip.sty}{%
    \usepackage{parskip}
  }{% else
    \setlength{\parindent}{0pt}
    \setlength{\parskip}{6pt plus 2pt minus 1pt}}
}{% if KOMA class
  \KOMAoptions{parskip=half}}
\makeatother
\usepackage{xcolor}
\IfFileExists{xurl.sty}{\usepackage{xurl}}{} % add URL line breaks if available
\IfFileExists{bookmark.sty}{\usepackage{bookmark}}{\usepackage{hyperref}}
\hypersetup{
  pdftitle={Hölder Spaces},
  hidelinks,
  pdfcreator={LaTeX via pandoc}}
\urlstyle{same} % disable monospaced font for URLs
\usepackage[margin=1in]{geometry}
\setlength{\emergencystretch}{3em} % prevent overfull lines
\providecommand{\tightlist}{%
  \setlength{\itemsep}{0pt}\setlength{\parskip}{0pt}}
\setcounter{secnumdepth}{-\maxdimen} % remove section numbering
\usepackage{mathrsfs}
\usepackage{amsthm}
\renewcommand{\square}{\hfill\qed}

\title{Hölder Spaces}
\author{Sean Richardson}
\date{}

\begin{document}
\maketitle

\hypertarget{motivation-via-arzeluxe0-ascoli}{%
\section{Motivation via
Arzelà-Ascoli}\label{motivation-via-arzeluxe0-ascoli}}

Suppose we have a sequence of functions \(u_n\) and wish to prove that
this sequence converges uniformly (we often encounter this, for example,
when solving PDEs). If we are working on a compact subset
\(X \subset \mathbb{R}^n\), the Arzelà-Ascoli theorem provides a
convenient criteria for uniform convergence: pointwise boundedness and
equicontinuity. Recall these are defined as follows, letting \(C(X)\)
denote the space of continuous functions \(X \to \mathbb{R}\).

\textbf{Def (pointwise boundedness).} A subset
\(\mathcal{F} \subset C(X)\) is \emph{pointwise bounded} if for all
\(x \in X\), the set \(\{u(x): u \in \mathcal{F}\} \subset \mathbb{R}\)
is bounded.

\textbf{Def (equicontinuity).} A subset \(\mathcal{F} \subset C(X)\) is
\emph{equicontinuous} if for all \(x \in X\) and \(\varepsilon > 0\),
there exists \(\delta > 0\) such that \(|x-y| < \delta\) implies we have
the bound \(|u(x) - u(y)| < \varepsilon\) for all \(u \in \mathcal{F}\).

Importantly, given such a \(x\) and \(\varepsilon\) as above, the same
\(\delta\) must work for all \(u \in \mathcal{F}\). Keep in mind the
subset \(\mathcal{F}\) above is often simply a sequence \(u_n\) of
continuous functions. The Arzelà-Ascoli theorem then promises

\textbf{Theorem (Arzelà-Ascoli).} If \(\mathcal{F} \subset C(X)\) is
equicontinuous and pointwise bounded, then there exists a subsequence
\(u_k \subset \mathcal{F}\) such that \(u_k \to u \in C(X)\) uniformly.

Equicontinuity is generally harder than pointwise boundedness to verify
when using Arzelà-Ascoli. A convenient sufficient condition for
equicontinuity is there exists a fixed \(C > 0\) so that \[
    |u(x) - u(y)| \leq C \|x-y\| \tag{1} \label{eq:lipshitz}
\] for all \(u \in \mathcal{F}\). This immediately implies
equicontinuity because for any \(\varepsilon > 0\) we can simply take
\(|x-y| < \varepsilon/C\). However, an even weaker sufficient condition
which is often easier to check is that for some fixed
\(\alpha \in (0,1]\) there exists \(C > 0\) such that \[
    |u(x) - u(y)| \leq C \|x-y\|^{\alpha} \tag{2} \label{eq:holder}
\] for all \(u \in \mathcal{F}\). In this case, we get equicontinuity by
taking \(\|x-y\| < (\varepsilon/C)^{1/\alpha}\) for any given
\(\varepsilon > 0\). If \(u\) satisfies (\ref{eq:lipshitz}), it is
called \emph{Lipshitz continuous} and if \(u\) satisfies
(\ref{eq:holder}), it is called \emph{\(\alpha\)-Hölder continuous}.
This terminology is due to the following.

\textbf{Proposition.} If \(u: X \to \mathbb{R}\) satisfies either (1) or
(2), it is a continuous function.

\emph{Proof.} Suppose \(u\) is \(\alpha\)-Hölder continuous with Hölder
constant \(C > 0\). Then or any \(\varepsilon > 0\), taking
\(\|x-y\| < (\varepsilon/C)^{1/\alpha}\) guarantees \[
    |u(x)-u(y)| \leq C|x-y|^{\alpha} < \varepsilon.
\]

\(\square\)

Given an \(\alpha\)-Hölder continuous function \(u: X \to \mathbb{R}\),
the minimum constant \(C\) is given by the \emph{\(\alpha\)-Hölder
semi-norm} \[
    [u]_{\alpha} := \sup_{x \neq y}\frac{|u(x) - u(y)|}{\|x-y\|^{\alpha}}.
\] This semi-norm is convenient because
\(\{ [u]_{\alpha}: u \in \mathcal{F}\}\) bounded implies \(\mathcal{F}\)
is equicontinuous. Additionally, note that if
\(\{ \|u\|_{\sup}: u \in \mathcal{F}\}\) is bounded, then
\(\mathcal{F}\) is pointwise bounded where \(\| \cdot \|_{\sup}\)
denotes the uniform norm. This motivates defining the
\emph{\(\alpha\)-Hölder norm} \[
    \|u\|_{\alpha} := \|u\|_{\sup} + [u]_{\alpha}
\] because if \(\{\|u\|_{\alpha}: u \in \mathcal{F}\}\) is bounded, then
\(\mathcal{F}\) is equicontinuous and pointwise bounded, so
Arzelà-Ascoli applies and there exists \(u_k \in \mathcal{F}\) such that
\(u_k \to u \in C(X)\) uniformly.

The point is the \(\alpha\)-Hölder norm \(\|\cdot\|_{\alpha}\) is useful
because it allows us to conclude uniform convergence due to
Arzelà-Ascoli. We can rephrase this oberservation using the language of
functional analysis, which begins by considering the function space of
\(\alpha\)-Hölder continuous functions.

\textbf{Def (Hölder space).} For a compact subset
\(X \subset \mathbb{R}^n\) and \(\alpha \in (0,1]\), the \emph{Hölder
space} \(C^{\alpha}(X)\) is \[
    C^{\alpha}(X) := \{u \in C(X) : \|u\|_{\alpha} < \infty\}.
\]

\end{document}
