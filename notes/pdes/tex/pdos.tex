% Options for packages loaded elsewhere
\PassOptionsToPackage{unicode}{hyperref}
\PassOptionsToPackage{hyphens}{url}
%
\documentclass[
]{article}
\usepackage{lmodern}
\usepackage{amssymb,amsmath}
\usepackage{ifxetex,ifluatex}
\ifnum 0\ifxetex 1\fi\ifluatex 1\fi=0 % if pdftex
  \usepackage[T1]{fontenc}
  \usepackage[utf8]{inputenc}
  \usepackage{textcomp} % provide euro and other symbols
\else % if luatex or xetex
  \usepackage{unicode-math}
  \defaultfontfeatures{Scale=MatchLowercase}
  \defaultfontfeatures[\rmfamily]{Ligatures=TeX,Scale=1}
\fi
% Use upquote if available, for straight quotes in verbatim environments
\IfFileExists{upquote.sty}{\usepackage{upquote}}{}
\IfFileExists{microtype.sty}{% use microtype if available
  \usepackage[]{microtype}
  \UseMicrotypeSet[protrusion]{basicmath} % disable protrusion for tt fonts
}{}
\makeatletter
\@ifundefined{KOMAClassName}{% if non-KOMA class
  \IfFileExists{parskip.sty}{%
    \usepackage{parskip}
  }{% else
    \setlength{\parindent}{0pt}
    \setlength{\parskip}{6pt plus 2pt minus 1pt}}
}{% if KOMA class
  \KOMAoptions{parskip=half}}
\makeatother
\usepackage{xcolor}
\IfFileExists{xurl.sty}{\usepackage{xurl}}{} % add URL line breaks if available
\IfFileExists{bookmark.sty}{\usepackage{bookmark}}{\usepackage{hyperref}}
\hypersetup{
  pdftitle={Pseudo-differential operators},
  hidelinks,
  pdfcreator={LaTeX via pandoc}}
\urlstyle{same} % disable monospaced font for URLs
\usepackage[margin=1in]{geometry}
\setlength{\emergencystretch}{3em} % prevent overfull lines
\providecommand{\tightlist}{%
  \setlength{\itemsep}{0pt}\setlength{\parskip}{0pt}}
\setcounter{secnumdepth}{-\maxdimen} % remove section numbering
\usepackage{mathrsfs}

\title{Pseudo-differential operators}
\author{Sean Richardson}
\date{}

\begin{document}
\maketitle

\hypertarget{bessel-potential}{%
\section{Bessel Potential}\label{bessel-potential}}

Consider the PDE \[
    (1-\Delta)u = f \quad\quad \text{in } \mathbb{R}^n.
\] We can take the Fourier transform of both sides and solve for the
Fourier transform \(\hat{u}(\xi)\) of the solution. \[
    (1+|\xi|^2)\hat{u}(\xi) = \hat{f}(\xi) \implies \hat{u}(\xi) = \frac{1}{1+|\xi|^2} \hat{f}(\xi).
\] Taking the inverse Fourier transform \(\mathcal{F}^{-1}\) of the
solution allows us to solve for the solution \[
    u(x) = \mathcal{F}^{-1}\left(\frac{1}{1+|\xi|^2}\hat{f}(\xi)\right).
\] In other words, this PDE can be solved with the \emph{Bessel
Potential} operator \[
    Bf := \mathcal{F}^{-1}\left(\frac{1}{1+|\xi|^2}\hat{f}(\xi)\right),
\] which inverts the differential operator \((1-\Delta)\). Such
``pseudodifferential operators'' appear frequently when trying to invert
differential operators.

\hypertarget{definition-of-pseudodifferential-operators}{%
\section{Definition of Pseudodifferential
Operators}\label{definition-of-pseudodifferential-operators}}

Consider some order \(m\) differential operator
\(P(x,D) = \sum_{|\alpha| \leq m} a_{\alpha}(x)D^{\alpha}\) on
\(\mathbb{R}^n\) with \(a_{\alpha}(x)\) smooth and all derivatives
\(\partial^{\beta}a_{\alpha}(x)\) bounded. Use that the Fourier
transform intertwines differentiation and multiplication to compute
\begin{align}
    P(x,D)u
    &= P(x,D)\mathcal{F}^{-1}(\hat{u}(\xi))
    = \sum_{|\alpha| \leq m} a_{\alpha}(x)D^{\alpha}\mathcal{F}^{-1}(\hat{u}(\xi))\\\
    &= \mathcal{F}^{-1}\left(\sum_{|\alpha| \leq m} a_{\alpha}(x)\xi^{\alpha}\hat{u}(\xi)\right)
    = \mathcal{F}^{-1}\left(p(x,\xi)\hat{u}(\xi)\right)
\end{align} where the polynomial
\(p(x,\xi) := \sum_{|\alpha| \leq m} a_{\alpha}(x) \xi^{\alpha}\) is the
\emph{(full) symbol}. The above gives \[
    P(x,D)u = \mathcal{F}^{-1}\left(p(x,\xi)\hat{u}(\xi)\right).
\] That is, applying a differential operator to a function is equivalent
to multiplying it's Fourier transform by the symbol. In other words, the
Fourier transorm intertwines the differential operator and this
principle symbol: \[
    P(x,D)\mathcal{F}^{-1} = \mathcal{F}^{-1}p(x,\xi) 
    \quad \text{or} \quad
    \mathcal{F} \circ P(x,D) = p(x,\xi) \mathcal{F}.
\] We saw above that the differential operator \((1-\Delta)\) has symbol
\(1+|\xi|^2\). The idea of pseudodifferential operators is to consider
operators with non-polynomial symbols \(p(x, \xi)\). For example, we
defined the Bessel potential operator above to have ``symbol''
\((1+|\xi|^2)^{-1}\). We ask the question then: what symbols can we use
to define a nice operator?

Because we are working with Fourier transforms, one particularly
important property of our differential operator \(P(x,D)\) is the
mapping property
\(P(x,D): \mathscr{S}(\mathbb{R}^n) \to \mathscr{S}(\mathbb{R}^n)\)
where \(\mathscr{S}(\mathbb{R}^n)\) is Schwartz space. This mapping
property is due to two properties of the symbol \(p(x, \xi)\).

Firstly, the symbol \(p(x,\xi) = \sum a_{\alpha}(x)\xi^{\alpha}\) is
smooth. This allows us to write the derivatives \[
    \partial_x^{\gamma}(P(x,D)u)
    = \partial_x^{\gamma} \mathcal{F}^{-1}(p(x, \xi))\hat{u}(\xi)
    = \int_{\mathbb{R}^n} \partial_x^{\gamma}(e^{ix \cdot \xi}p(x, \xi))\hat{u}(\xi)d\xi
\] in terms of the derivatives
\(\partial_x^{\gamma}(e^{ix \cdot \xi}p(x, \xi))\), which after
expanding by product rule relies on the derivatives
\(\partial_x^{\alpha} p_x(x,\xi)\) of the symbol.

Secondly, the above integrals are well-defined because for differential
operators with symbol \(p(x, \xi)\), if
\(\hat{u}(\xi) \in \mathscr{S}(\mathbb{R}^n)\), then
\(\partial^{\alpha}p(x,\xi) \hat{u}(\xi) \in \mathscr{S}(\mathbb{R}^n)\).
To see this, compute the bound \begin{align}
    |\partial_{\xi}^{\beta} \partial_{x}^{\alpha} p(x,\xi)|
    &= \left|\partial_{\xi}^{\beta} \partial_{x}^{\alpha} \sum_{|\mu| \leq m} a_{\mu}(x)\xi^{\mu}\right|\\\
    &\leq \sum_{|\mu| \leq m} |\partial_x^{\alpha}a_{\mu}(x)|\frac{\mu!}{\beta!}|\xi|^{\mu-\beta}
    \leq C\sum_{|\mu| \leq m}|\xi|^{\mu - \beta}
    \leq C_{\alpha, \beta} \langle \xi \rangle^{m - |\beta|}.
\end{align} where \(\langle \xi \rangle^2 = (1+|\xi|^2)\) is the bracket
notation and we used that \(\partial_x^{\alpha}a_{\mu}\) are bounded.
Thus two key properties of the symbol \(p(x,\xi)\) of a differential
operator are smoothness, and that we have the bound
\(|\partial_{\xi}^{\beta} \partial_{x}^{\alpha} p(x,\xi)| \leq C_{\alpha, \beta} \langle \xi \rangle^{m - |\beta|}\).
We then consider a more general set of symbols \(p(x, \xi)\) that
satisfy precisely these two properties as well as the corresponding
class of operators.

\textbf{Def (Symbol class \(S^m\).)} Given some \(m \in \mathbb{Z}\),
define the \emph{symbol class} \(S^m\) of order \(m\) to be all
\(p \in C^{\infty}(\mathbb{R}_x^{n} \times \mathbb{R}^n_{\xi})\) so that
for all \(\alpha, \beta\) there exists \(C_{\alpha, \beta} > 0\) so that
\[
    |\partial_{\xi}^{\beta}\partial_{x}^{\alpha}p(x, \xi)| \leq C_{\alpha, \beta}\langle \xi \rangle^{m-|\beta|}.
\]

\textbf{Def (Pseudodifferential operator).} For some \(p \in S^m\), we
define its \emph{quantization} \(P = \operatorname{Op}(p)\) to be the
operator \[
    Pf(x) := (2\pi)^{-1}\int_{\mathbb{R}^n} e^{ix \cdot \xi} p(x, \xi) \hat{f}(\xi)d\xi.
\] This operator \(P\) is called a \emph{pseudodifferential operator of
order \(m\)} and the set of all pseudodifferential operators of order
\(m\) is denoted \(\Psi^m = \{\operatorname\{Op\}(p): p \in S^m\}.\)

Later we may consider an alternate bound on the derivatives of the
symbols, but the symbol class as defined above is the most common and
most closely resembles the bound on differential operators, which
results in nice Sobolev mapping properties. Note that the definitions
above immediately generalize to \(m \in \mathbb{R}\).

\end{document}
