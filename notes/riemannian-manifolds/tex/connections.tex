% Options for packages loaded elsewhere
\PassOptionsToPackage{unicode}{hyperref}
\PassOptionsToPackage{hyphens}{url}
%
\documentclass[
]{article}
\usepackage{lmodern}
\usepackage{amssymb,amsmath}
\usepackage{ifxetex,ifluatex}
\ifnum 0\ifxetex 1\fi\ifluatex 1\fi=0 % if pdftex
  \usepackage[T1]{fontenc}
  \usepackage[utf8]{inputenc}
  \usepackage{textcomp} % provide euro and other symbols
\else % if luatex or xetex
  \usepackage{unicode-math}
  \defaultfontfeatures{Scale=MatchLowercase}
  \defaultfontfeatures[\rmfamily]{Ligatures=TeX,Scale=1}
\fi
% Use upquote if available, for straight quotes in verbatim environments
\IfFileExists{upquote.sty}{\usepackage{upquote}}{}
\IfFileExists{microtype.sty}{% use microtype if available
  \usepackage[]{microtype}
  \UseMicrotypeSet[protrusion]{basicmath} % disable protrusion for tt fonts
}{}
\makeatletter
\@ifundefined{KOMAClassName}{% if non-KOMA class
  \IfFileExists{parskip.sty}{%
    \usepackage{parskip}
  }{% else
    \setlength{\parindent}{0pt}
    \setlength{\parskip}{6pt plus 2pt minus 1pt}}
}{% if KOMA class
  \KOMAoptions{parskip=half}}
\makeatother
\usepackage{xcolor}
\IfFileExists{xurl.sty}{\usepackage{xurl}}{} % add URL line breaks if available
\IfFileExists{bookmark.sty}{\usepackage{bookmark}}{\usepackage{hyperref}}
\hypersetup{
  pdftitle={Covariant Derivative},
  hidelinks,
  pdfcreator={LaTeX via pandoc}}
\urlstyle{same} % disable monospaced font for URLs
\usepackage[margin=1in]{geometry}
\setlength{\emergencystretch}{3em} % prevent overfull lines
\providecommand{\tightlist}{%
  \setlength{\itemsep}{0pt}\setlength{\parskip}{0pt}}
\setcounter{secnumdepth}{-\maxdimen} % remove section numbering
\usepackage{mathrsfs}
\usepackage{amsthm}
\renewcommand{\square}{\hfill\qed}

\title{Covariant Derivative}
\author{Sean Richardson}
\date{}

\begin{document}
\maketitle

\hypertarget{covariant-derivative}{%
\section{Covariant Derivative}\label{covariant-derivative}}

\hypertarget{differentiating-vector-fields-in-mathbbrn}{%
\subsection{\texorpdfstring{Differentiating vector fields in
\(\mathbb{R}^n\)}{Differentiating vector fields in \textbackslash mathbb\{R\}\^{}n}}\label{differentiating-vector-fields-in-mathbbrn}}

Given a path \(\gamma(t)\) in \(\mathbb{R}^n\), we know we can represent
its velocity by \[
    \dot{\gamma}(t) = \dot{\gamma}^i(t) \partial_i.
\] Furthermore, in the case of \(\mathbb{R}^n\), we understand the
acceleration of the curve \(\gamma(t)\) to be given by \[
    \ddot{\gamma}(t) = \frac{d }{d t}\left(\dot{\gamma}^i(t) \partial_i\right) = \ddot{\gamma}^i(t) \partial_i.
\] In computing the acceleration, we are differentiating a vector field.
More generally in \(\mathbb{R}^n\), we understand the derivative of some
vector field \(X\) in the direction \(v \in T_p\mathbb{R}^n\) at point
\(p\), which we denote \(\nabla_v X\), to be \[
    \nabla_v X = \left.\frac{d }{d t}\right|_{t=0}X(\gamma(t)) = \left.\frac{d }{d t}\right|_{t=0}X^i(\gamma(t))\partial_i = v(X^i)\partial_i.
\] where \(\gamma(t)\) is a smooth curve such that
\(\dot{\gamma}(0) = v\). This directional derivative enjoys the
following nice properties:

\begin{enumerate}
\def\labelenumi{\arabic{enumi}.}
\tightlist
\item
  First, the directional derivative is linear with respect to the
  direction. Indeed, for any vectors \(v,w\) based at point
  \(p \in \mathbb{R}^n\), real numbers \(a,b\), and vector field \(X\)
  we find \[
   \nabla_{av + bw}X
   = (av + bw)X^i \partial_i
   = avX^i\partial_i + bwX^i\partial_i
   = a\nabla_vX + b\nabla_wX.
  \]
\item
  Second, this directional derivative is linear with respect to the
  vector fields. Indeed, for any vector \(v\) based at
  \(p \in \mathbb{R}^n\), any real numbers \(a,b\), and any vector
  fields \(X,Y\) that \[
   \nabla_v(aX + bY)
   = v(aX^i + bY^i)\partial_i
   = avX^i\partial_i + bvY^i\partial_i
   = a\nabla_vX + b\nabla_vY
  \]
\item
  Finally, we have a product rule. For any vector \(v\) at point
  \(p \in \mathbb{R}^n\), vector field \(X\), and smooth function \(f\)
  we find \[
   \nabla_v(fX)
   = v(fX^i)\partial_i
   = (vf)X^i\partial_i + f(vX^i)\partial_i
   = (vf)X + f\nabla_vX
  \]
\end{enumerate}

\hypertarget{the-problem-with-differentiating-vector-fields-on-manifolds}{%
\subsection{The problem with differentiating vector fields on
manifolds}\label{the-problem-with-differentiating-vector-fields-on-manifolds}}

However, the definition of the directional derivative of a vector field
\(X\) in direction \(v\) \[
    \left.\frac{d }{d t}\right|_{t=0} X(\gamma(t)) = \lim_{t \to 0}\frac{X(\gamma(t)) - X(\gamma(0))}{t}
\] does not make sense on a general Riemannian manifold! The vectors
\(X(\gamma(t))\) and \(X(\gamma(0))\) belong to the two different
tangent spaces \(T_{\gamma(t)}M\) and \(T_{\gamma(0)}M\), so it does not
make sense to subtract them. Note that in the case of \(\mathbb{R}^n\),
there is a natural identification between different tangent spaces
\(T_p\mathbb{R}^n\) and \(T_q\mathbb{R}^n\) by simply translating the
vectors. However, this a consequence of having a nice coordinate frame
\((\partial_i)\), for we are naturally identifying
\(v^i\partial_i \in T_p\mathbb{R}^n\) with
\(v^i\partial_i \in T_q\mathbb{R}^n\).

\hypertarget{connections}{%
\subsection{Connections}\label{connections}}

We are looking to define a directional derivative operation
\(\nabla_v X\) on a general smooth manifold. Formally, we are looking
for a map \(\nabla: T_pM \times \Gamma(TM) \to T_pM\) varying smoothly
with \(p\). If this map is denoted \(\nabla: (v, X) \mapsto \nabla_vX\),
we want \(\nabla\) to satisfy the following three properties we expect
from a directional derivative. In the following, \(a,b \in \mathbb{R}\)
are real numbers, \(v,w \in T_pM\) are vectors based at \(p\), and
\(X,Y \in \Gamma(TM)\) are vector fields on the manifold.

\begin{enumerate}
\def\labelenumi{\arabic{enumi}.}
\tightlist
\item
  Linearity with respect to the direction: \[
   \nabla_{av + bw}X = a\nabla_vX + b\nabla_wX.
  \]
\item
  Linearity with respect to the vector field: \[
   \nabla_v(aX + bY) = a\nabla_vX + b\nabla_vY.
  \]
\item
  Product rule: \[
   \nabla_v(fX) = (vf)X + f\nabla_vX.
  \]
\end{enumerate}

Such a map \(\nabla: T_pM \times \Gamma(TM) \to T_pM\) varying smoothly
with \(p\) that satisfies the three properties above is called a
\emph{connection}. Equivalently, we can consider two vector fields
\(X,Y\) and say \(\nabla_X Y\) is vector field such that
\((\nabla_X Y)(p) = \nabla_{X(p)} Y\) so that we now have a map
\(\nabla: \Gamma(TM) \times \Gamma(TM) \to \Gamma(TM)\). The directional
derivative \((\nabla_X Y)\) corresponding to a connection is called the
\emph{covariant derivative} of \(Y\) in the direction of \(X\). The main
question, however, is how do we choose which connection to use? In
general, there are many connections on a smooth manifold: choose any
coordinate frame \((\partial_i)\) and decide the derivative of each
coordinate frame in the direction of all the other coordinate frames at
every point. That is, choose functions \(A_{ij}^k\) such that \[
    \nabla_{\partial_i}\partial_j = A_{ij}^k \partial_k.
\] Then given any arbitrary vector fields \(X = X^i\partial_i\) and
\(Y = Y^j\partial_j\), we can determine what the covariant derivative
\(\nabla_XY\) should be by \[
    \nabla_X Y 
    = \nabla_{X^i \partial_i}(Y^j \partial_j)
    = X^i((\partial_iY^j)\partial_j + Y^j \nabla_{\partial_i}\partial_j)
    = X^i(\partial_iY^k + Y^jA_{ij}^k)\partial_k.
    \label{eq:As}
    \tag{A}
\] For any smooth functions \(A_{ij}^k\) that we choose, the above
formula in fact defines a connection.

\textbf{Exercise.} Verify that for any smooth functions \(A_{ij}^k\),
the formula (\ref{eq:As}) defines a connection by checking the three
necessary properties are satisfied.

Thus there are many possible connections on a general smooth manifold.
For some intuition for connections, note that choosing a connection
gives us a sense of acceleration on our manifold. Given a curve
\(\gamma(t)\), it's velocity at each point along the curve is given by
the tangent vectors \(\dot{\gamma}(t)\). Then the acceleration should be
the change of \(\dot{\gamma}(t)\) in direction \(\dot{\gamma}(t)\). That
is, the \emph{acceleration} of \(\gamma(t)\) is defined to be
\(\ddot{\gamma}(t) = \nabla_{\dot{\gamma}(t)}\dot{\gamma}(t)\).

\end{document}
