% Options for packages loaded elsewhere
\PassOptionsToPackage{unicode}{hyperref}
\PassOptionsToPackage{hyphens}{url}
%
\documentclass[
]{article}
\usepackage{lmodern}
\usepackage{amssymb,amsmath}
\usepackage{ifxetex,ifluatex}
\ifnum 0\ifxetex 1\fi\ifluatex 1\fi=0 % if pdftex
  \usepackage[T1]{fontenc}
  \usepackage[utf8]{inputenc}
  \usepackage{textcomp} % provide euro and other symbols
\else % if luatex or xetex
  \usepackage{unicode-math}
  \defaultfontfeatures{Scale=MatchLowercase}
  \defaultfontfeatures[\rmfamily]{Ligatures=TeX,Scale=1}
\fi
% Use upquote if available, for straight quotes in verbatim environments
\IfFileExists{upquote.sty}{\usepackage{upquote}}{}
\IfFileExists{microtype.sty}{% use microtype if available
  \usepackage[]{microtype}
  \UseMicrotypeSet[protrusion]{basicmath} % disable protrusion for tt fonts
}{}
\makeatletter
\@ifundefined{KOMAClassName}{% if non-KOMA class
  \IfFileExists{parskip.sty}{%
    \usepackage{parskip}
  }{% else
    \setlength{\parindent}{0pt}
    \setlength{\parskip}{6pt plus 2pt minus 1pt}}
}{% if KOMA class
  \KOMAoptions{parskip=half}}
\makeatother
\usepackage{xcolor}
\IfFileExists{xurl.sty}{\usepackage{xurl}}{} % add URL line breaks if available
\IfFileExists{bookmark.sty}{\usepackage{bookmark}}{\usepackage{hyperref}}
\hypersetup{
  pdftitle={Levi-Civita Connection},
  hidelinks,
  pdfcreator={LaTeX via pandoc}}
\urlstyle{same} % disable monospaced font for URLs
\usepackage[margin=1in]{geometry}
\setlength{\emergencystretch}{3em} % prevent overfull lines
\providecommand{\tightlist}{%
  \setlength{\itemsep}{0pt}\setlength{\parskip}{0pt}}
\setcounter{secnumdepth}{-\maxdimen} % remove section numbering
\usepackage{mathrsfs}
\usepackage{amsthm}
\renewcommand{\square}{\hfill\qed}

\title{Levi-Civita Connection}
\author{Sean Richardson}
\date{}

\begin{document}
\maketitle

\hypertarget{levi-civita-connection}{%
\section{Levi-Civita Connection}\label{levi-civita-connection}}

On a Riemannian manifold, there is a natural choice for which connection
to use. Choosing a connection is choosing a sense of acceleration on our
manifold. For a Riemannian manifold \(M\), a natural choice is to agree
that geodesics have \(0\) accelertaion. Indeed, geodesics are paths that
go in a ``straight line'' without changing velocity. Thus we would like
a connection \(\nabla\) such that for any geodesic \(\gamma(t)\) we have
\(\nabla_{\dot{\gamma}(t)}\dot{\gamma}(t) = 0\). If we have
\(\gamma(t) = (x^1(t), \cdots, x^n(t))\) in local coordinates, this
requirement is equivalent to \begin{align}
    0 
    &= \nabla_{\dot{\gamma}(t)} \dot{\gamma}(t)\\\
    &= \nabla_{\dot{\gamma}(t)} (\dot{x}^j(t)\partial_j)\\\
    &= \dot{\gamma}(t)(\dot{x}^j(t))\partial_j + \dot{x}^j(t)\nabla_{\dot{\gamma}(t)} \partial_j\\\
    &= \ddot{x}^j(t)\partial_j + \dot{x}^j(t)\nabla_{\dot{x}^i\partial_i} \partial_j\\\
    &=  (\ddot{x}^k(t) + \dot{x}^i\dot{x}^j(t)A_{ij}^k) \partial_k.
\end{align} That is, this is equivalent to \[
    \ddot{x}^k(t) + \dot{x}^i\dot{x}^j(t)A_{ij}^k = 0 \quad \text{for all } k.
\] Thus we want to choose our functions \(A_{ij}^k\) so that all
geodesics satisfy the above equation. However, comparing the above
equation to the geodesic equation, we realize that this will always be
true if we choose \(A_{ij}^k = \Gamma_{ij}^k\) to be the Christoffel
symbols! Therefore, on a Riemannian manifold we should define our
connection in coordinates by setting \[
    \nabla_{\partial_i}\partial_j = \Gamma_{ij}^k \partial_k.
    \tag{C}
    \label{eq:lc-coords}
\] This definition, however, opens one important question: does this
choice of connection rely on the choice of coordinates? The answer is
no, which we will prove by later defining this natural connection in a
coordinate independent way. The resulting connection is called the
``Levi-Civita connection''.

Recall the connection for Euclidean space \(\mathbb{R}^n\) is given by
\[
    \nabla_{X}Y = XY^i\partial_i.
\] This \emph{Euclidean connection} satisfies the product rule \[
    \nabla_{Z}\langle X , Y\rangle
    = \nabla_{X} \left(\sum_{i}X^iY^i\right)
    = \sum_{i}(XZ^i)Y^i + X^i(ZY^i)
    = \langle \nabla_Z X , Y\rangle + \langle X , \nabla_ZY\rangle.
\] Additionally, observe that for any vector fields \(X,Y\) over
\(\mathbb{R}^n\), we can compute the difference of the Euclidean
connection \(\nabla_X Y - \nabla_Y X\) to be \[
   \nabla_X Y - \nabla_Y X
   = XY^i\partial_i - YX^i\partial_i
   = (XY^i - YX^i) \partial_i. 
\] But this is juts the Lie bracket \([X,Y]\). That is, the Euclidean
connection satisfies the commutator relation \[
    \nabla_X Y - \nabla_Y X = [X,Y].
    \tag{S}
    \label{eq:symmetry}
\]

It turns out that for a Riemannian manifold \(M\), the Levi-Civita
connection defined in local coordinates by (\ref{eq:lc-coords}) also
satisfies the commutator relation (\ref{eq:symmetry}) as well as the
following product rule with respect to the metric \(g\). \[
    \nabla_Z\langle X , Y\rangle_g = \langle \nabla_Z X , Y\rangle_g + \langle X , \nabla_Z Y\rangle_g.
    \tag{M}
    \label{eq:compatibility}
\] A connection satisfying (\ref{eq:symmetry}) is called
\emph{symmetric} and a connection satisfying the product rule
(\ref{eq:compatibility}) is said to be \emph{compatible with the
metric}. We begin by showing the symmetry which is simply a coordinate
computation.

\textbf{Prop.} The Levi-Civita connection as defined in coordinates by
(\ref{eq:lc-coords}) is symmetric.

\emph{Proof.} Compute \begin{align}
    \nabla_X Y - \nabla_Y X
    &= \nabla_{X^i \partial_i}(Y^j \partial_j) - \nabla_{Y^j\partial_j}(X^i \partial_i)\\\
    &= X^i((\partial_iY^j)\partial_j + Y^j \nabla_{\partial_j}\partial_i)
    -  Y^j((\partial_jX^i)\partial_i + X^i \nabla_{\partial_i}\partial_j)\\\
    &= (XY^j\partial_j - YX^i\partial_i) + X^iY^j(\nabla_{\partial_j}\partial_i - \nabla_{\partial_i}\partial_j)\partial_k\\\
    &= [X,Y] + X^iY^j(\Gamma_{ij}^k - \Gamma_{ji}^k)\partial_k.
\end{align} Then the result follows from
\(\Gamma_{ij}^k = \Gamma_{ji}^k\) which we can see from the definition
of Christoffel symbols: \[
    \Gamma_{ij}^k = \frac{1}{2}g^{kl}(\partial_ig_{jl} + \partial_jg_{li} - \partial_lg_{ij})
\]

\(\square\)

Next we show the Levi-Civita connection as defined in coordinates is
compatible with the metric, which follows from a substantially longer
coordinate computation.

\textbf{Prop.} The Levi-Civita connection as defined in coordinates by
(\ref{eq:lc-coords}) is compatible with the metric.

\emph{Proof.} First we expand out the right side of
(\ref{eq:compatibility}). \begin{align}
    \langle \nabla_Z X , Y\rangle + \langle X , \nabla_Z Y\rangle
    &= \langle \nabla_{Z^k \partial_k}(X^i \partial_i) , Y^j\partial_j \rangle
    + \langle X^i \partial_i , \nabla_{Z^k \partial_k}(Y^j \partial_j)\rangle\\\
    &= Z^k(Y^j\langle \partial_k X^i\partial_i + X^i \nabla_{\partial_k}\partial
    _i , \partial_j\rangle
    + X^i\langle \partial_i , \partial_kY^j \partial_j + Y^j \nabla_{\partial_k}\partial_j\rangle)\\\
    &= Z^k(Y^j\langle (\partial_kX^l + X^i\Gamma_{ik}^l)\partial_l , \partial_j\rangle
    + X^i \langle \partial_i , (\partial_kY^l + Y^j\Gamma_{kj}^l)\partial_l\rangle)\\\
    &= Z^kY^j(\partial_kX^l + X^i\Gamma_{ik}^l)g_{lj}
    + Z^kX^i(\partial_kY^l + Y^j\Gamma_{kj}^l)g_{il}\\\
    &= Z^k(Y^j\partial_kX^lg_{lj} + X^i\partial_kY^l g_{il}) + Z^kX^iY^j(\Gamma_{ik}^lg_{lj} + \Gamma_{kj}^lg_{il}).
\end{align} Next we expand out the left size of
(\ref{eq:compatibility}). \[
    \nabla_Z\langle X , Y\rangle
    = Z^k \partial_k\langle X^i \partial_i , Y^j \partial_j\rangle
    = Z^k \partial_k(X^i Y^j g_{ij})
    = Z^k(Y^j\partial_kX^ig_{ij} + X^i\partial_kY^jg_{ij})
    + Z^kX^iY^j\partial_kg_{ij}.
\] Note these expansions are quite similar, and we see that in fact the
right and left sides of (\ref{eq:compatibility}) are equal so long as we
can show \[
    \partial_kg_{ij} = \Gamma_{ik}^lg_{lj} + \Gamma_{kj}^lg_{il}.
\] Indeed, to show this we use the definition of the Christoffel symbols
\[
    \Gamma_{ij}^k = \frac{1}{2}g^{kl}(\partial_ig_{jl} + \partial_jg_{li} - \partial_lg_{ij})
\] and apply the matrix \(g_{km}\) to both sides to conclude \[
    \Gamma_{ij}^k g_{km} = \frac{1}{2}(\partial_ig_{jm} + \partial_jg_{mi} - \partial_mg_{ij}).
\] Thus using the above expression twice we can compute \[
    \Gamma_{ik}^lg_{lj} + \Gamma_{kj}^lg_{il}
    = \frac{1}{2}(\partial_ig_{kj} + \partial_kg_{ji} - \partial_jg_{ik})
    + \frac{1}{2}(\partial_kg_{ji} + \partial_jg_{ik} - \partial_ig_{kj})
    = \partial_kg_{ij}
\] as needed.

\(\square\)

It turns out that these two properties -- symmetry and metric
compatibility -- are quite special. In fact, on a Riemannian manifold
there will only be one connection that satisfies both properties.

\textbf{Prop. (Fundamental Theorem of Riemannian Geometry).} For any
Riemannian manifold \(M\), there exists a unique connection \(\nabla\)
that is both symmetric and compatible with the metric. This connection
is called the \emph{Levi-Civita connection}.

\emph{Proof in coordinates.} We have already demonstrated existence, for
the Levi-Civita connection is symmetric and metric-compatible. To see
why an arbitrary symmetric and metric-compatible connection \(\nabla\)
must be the Levi-Civita connection, we work locally in coordinates
\((x^i)\) and write \(\nabla_{\partial_i}\partial_j = A_{ij}^k\). By the
same computation we performed to show symmetry of the Levi-Civita
connection, we see the symmetry of \(\nabla\) is equivalent to
\(A_{ij}^k = A_{ji}^k\). Similarly, we see \(\nabla\) is compatible with
the metric exactly when \[
   \partial_kg_{ij} = A_{ik}^lg_{lj} + A_{kj}^lg_{il}.
\] by the corresponding computation for the Levi-Civita connection; this
expression is often called the \emph{first Christoffel identity}. These
two requirements give a linear system of \(\frac{1}{2}n^2(n+1)\)
equations with the same amount of unknowns. The trick to solve this
system is to permute the first Christoffel identity to get cancellation
and solve for the sum \begin{align}
    \partial_i g_{jl} + \partial_j g_{il} - \partial_l g_{ij}
    = (A_{ij}^p g_{pl} + A_{il}^p g_{jp}) + (A_{ji}^p g_{pl} + A_{jl}^p g_{ip}) - (A^p_{li} g_{pj} + A_{lj}^p g_{ip})
    = 2A_{ij}^p g_{pl}.
\end{align} Then applying the inverse matrix \(g^{kl}\) we recover the
definition of the Christoffel symbols: \[
    A_{ij}^k = \frac{1}{2}g^{kl}(\partial_i g_{jl} + \partial_j g_{il} - \partial_l g_{ij}).
\]

\(\square\)

\emph{Proof without coordinates.} Existence follows from the Levi-Civita
connection. For uniqueness, suppose \(\nabla\) is a symmetric and
metric-compatible connection and use both properties to write
\begin{align}
    X\langle Y , Z\rangle_g 
    = \langle \nabla_X Y , Z \rangle_g + \langle Y , \nabla_X Z \rangle_g
    = \langle \nabla_X Y , Z \rangle_g + \langle Y , \nabla_Z X \rangle_g + \langle Y , [X, Z]\rangle_g.
\end{align} We will use a similar trick as the proof in coordinates to
find an expression for \(\nabla\). By cyclically permuting the above, we
get two more identities: \begin{align}
    Y\langle Z , X\rangle_g 
    = \langle \nabla_Y Z , X \rangle_g + \langle Z , \nabla_Y X \rangle_g
    = \langle \nabla_Y Z , X \rangle_g + \langle Z , \nabla_X Y \rangle_g + \langle Z , [Y, X]\rangle_g\\\
    Z\langle X , Y\rangle_g 
    = \langle \nabla_Z X , Y \rangle_g + \langle X , \nabla_Z Y \rangle_g
    = \langle \nabla_Z X , Y \rangle_g + \langle X , \nabla_Y Z \rangle_g + \langle X , [Z, Y]\rangle_g.
\end{align} Now adding the first two equations and subtracting the third
gives the cancellation \begin{align}
X\langle Y , Z\rangle_g  + Y\langle Z , X\rangle_g - Z\langle X , Y\rangle_g
= 2 \langle \nabla_X Y , Z \rangle_g + \langle Y , [X, Z]\rangle_g + \langle Z , [Y, X]\rangle_g - \langle X , [Z, Y]\rangle_g.
\end{align} Thus we can solve for \(\langle \nabla_X Y , Z \rangle_g\)
to find \begin{align}
\langle \nabla_X Y , Z \rangle_g
= \frac{1}{2}(X\langle Y , Z\rangle_g  + Y\langle Z , X\rangle_g - Z\langle X , Y\rangle_g - \langle Y , [X, Z]\rangle_g - \langle Z , [Y, X]\rangle_g + \langle X , [Z, Y]\rangle_g).
\end{align} which uniquely determines the connection \(\nabla\). The
above is thus a coordinate-invariant expression for the Levi-Civita
connection and is called \emph{Koszul's formula}.

\end{document}
