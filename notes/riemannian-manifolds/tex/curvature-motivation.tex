% Options for packages loaded elsewhere
\PassOptionsToPackage{unicode}{hyperref}
\PassOptionsToPackage{hyphens}{url}
%
\documentclass[
]{article}
\usepackage{lmodern}
\usepackage{amssymb,amsmath}
\usepackage{ifxetex,ifluatex}
\ifnum 0\ifxetex 1\fi\ifluatex 1\fi=0 % if pdftex
  \usepackage[T1]{fontenc}
  \usepackage[utf8]{inputenc}
  \usepackage{textcomp} % provide euro and other symbols
\else % if luatex or xetex
  \usepackage{unicode-math}
  \defaultfontfeatures{Scale=MatchLowercase}
  \defaultfontfeatures[\rmfamily]{Ligatures=TeX,Scale=1}
\fi
% Use upquote if available, for straight quotes in verbatim environments
\IfFileExists{upquote.sty}{\usepackage{upquote}}{}
\IfFileExists{microtype.sty}{% use microtype if available
  \usepackage[]{microtype}
  \UseMicrotypeSet[protrusion]{basicmath} % disable protrusion for tt fonts
}{}
\makeatletter
\@ifundefined{KOMAClassName}{% if non-KOMA class
  \IfFileExists{parskip.sty}{%
    \usepackage{parskip}
  }{% else
    \setlength{\parindent}{0pt}
    \setlength{\parskip}{6pt plus 2pt minus 1pt}}
}{% if KOMA class
  \KOMAoptions{parskip=half}}
\makeatother
\usepackage{xcolor}
\IfFileExists{xurl.sty}{\usepackage{xurl}}{} % add URL line breaks if available
\IfFileExists{bookmark.sty}{\usepackage{bookmark}}{\usepackage{hyperref}}
\hypersetup{
  pdftitle={Curvature Motivation},
  hidelinks,
  pdfcreator={LaTeX via pandoc}}
\urlstyle{same} % disable monospaced font for URLs
\usepackage[margin=1in]{geometry}
\setlength{\emergencystretch}{3em} % prevent overfull lines
\providecommand{\tightlist}{%
  \setlength{\itemsep}{0pt}\setlength{\parskip}{0pt}}
\setcounter{secnumdepth}{-\maxdimen} % remove section numbering
\usepackage{mathrsfs}
\usepackage{amsthm}
\renewcommand{\square}{\hfill\qed}

\title{Curvature Motivation}
\author{Sean Richardson}
\date{}

\begin{document}
\maketitle

\hypertarget{motivation-for-the-riemann-curvature-tensor}{%
\section{Motivation for the Riemann Curvature
Tensor}\label{motivation-for-the-riemann-curvature-tensor}}

\hypertarget{studying-isometries}{%
\subsection{Studying Isometries}\label{studying-isometries}}

An important questions in Riemannian geometry is: given two Riemannian
manifolds, are they isometric? A first step to answering this extremely
hard problem is to consider the following local question: given two
different coordinate neighborhoods of Riemannian manifolds, are they
isometric? This is still extremely hard, but a more reasonable place to
start is: when does a point on a Riemannian manifold have a \emph{flat
neighborhood} (a neighborhood isometric to a neighborhood of
\(\mathbb{R}^n\))? This question was answered by Riemann and the
investigation of this problem naturally gives rise to the \emph{Riemann
curvature tensor}, which precisely encodes the extent to which a
manifold fails to be flat at every point.

\hypertarget{noncommutativity-of-parallel-transport}{%
\subsection{(Non)commutativity of Parallel
Transport}\label{noncommutativity-of-parallel-transport}}

First, notice that parallel transport in \(\mathbb{R}^n\) is path
independent: for example, transporting a vector along the \(x^1\) axis,
then the \(x^2\) axis will yield the same vector as transporting along
\(x^2\), then \(x^1\). Thus path independence of parallel transport is a
necessary condition for a neighborhood to be flat. Importantly,
path-independence of parallel transport is also sufficient, for it
allows us to construct flat coordinates as follows. Begin with an
orthornormal basis \(\{e_i\}\) of \(T_pM\), then if parallel transport
is path-independent in a neighborhood \(U\) of \(p\), transporting this
basis along any choice of paths yields the same frame \(E_i\) in \(U\),
implying each \(E_i\) is parallel in \emph{every direction}. Because
parallel transport preserves inner product, this frame is orthornormal.
We only need to check that \(E_i\) is a coordinate frame, which occurs
precisely when \(E_i\) is a commuting frame; indeed, because each
\(E_i\) is parallel in every direction, we have \(\nabla_X E_i = 0\) for
any \(X\) and so we compute \[
    [E_i, E_j] = \nabla_{E_i}E_j - \nabla_{E_j}E_i = 0 - 0 = 0.
\]

Thus we study the path-dependence of parallel transport. We quantify
this by investigating the noncommutativity of parallel transport around
small squares. Consider the square \(I \times I \to M\) parametrized by
\((s,t)\) such that \(p\) corresponds to \((0,0)\) and choose some
tangent vector \(v \in T_p M\). Then we compare the following two
methods of transporting \(v\) to the point \((\varepsilon, \delta)\). We
could first transport \(v\) in the \(s\) direction from \((0,0)\) to
\((\varepsilon, 0)\), which we denote
\(P^{(0,0)}_{(\varepsilon,0)}v \in T_{(\varepsilon,0)}\), then transport
this vector in the \(t\) direction from \((\varepsilon, 0)\) to
\((\varepsilon, \delta)\) to get
\(P_{(\varepsilon,\delta)}^{(\varepsilon,0)}P^{(0,0)}_{(\varepsilon,0)}v \in T_{(\varepsilon,\delta)}M\).
Alternatively, we could first transport \(v\) in the \(t\) direction
from \((0,0)\) to \((0, \delta)\) yielding
\(P^{(0,0)}_{(0,\delta)}v \in T_{(0,\delta)}\), then transport this
vector in the \(s\) direction from \((0, \delta)\) to
\((\varepsilon, \delta)\) resulting in
\(P_{(\varepsilon,\delta)}^{(0,\delta)}P^{(0,0)}_{(0,\delta)}v \in T_{(\varepsilon,\delta)}M\).
The difference between these two methods is quantified by

\[
    P_{(\varepsilon,\delta)}^{(\varepsilon,0)} P_{(\varepsilon,0)}^{(0,0)} v - P_{(\varepsilon,\delta)}^{(0,\delta)} P_{(0,\delta)}^{(0,0)} v.
\] We now compute a Taylor approximation of this difference. One trick
to simplify this computation is to extend \(v\) to any vector field
\(X\), then compare the value of \(X\) to the parallel transport of
\(v\) along each of the four segments, giving \begin{align}
    &P_{(\varepsilon,\delta)}^{(\varepsilon,0)} P_{(\varepsilon,0)}^{(0,0)} X_{(0,0)} - P_{(\varepsilon,\delta)}^{(0,\delta)} P_{(0,\delta)}^{(0,0)} X_{(0,0)}.\\\
    &= (P_{(\varepsilon,\delta)}^{(\varepsilon,0)}(P_{(\varepsilon,0)}^{(0,0)}X_{(0,0)} - X_{(\varepsilon,0)})
    - (P_{(\varepsilon,\delta)}^{(0,\delta)}X_{(0,\delta)} - X_{(\varepsilon,\delta)}))\\\
    &- (P_{(\varepsilon,\delta)}^{(0,\delta)}(P_{(0,\delta)}^{(0,0)}X_{(0,0)} - X_{(0,\delta)})
    - (P_{(\varepsilon,\delta)}^{(\varepsilon,0)}X_{(\varepsilon,0)} - X_{(\varepsilon,\delta)})).
\end{align} Recall that the parallel transport \(P_t\) of a vector field
\(X\) from \(p\) to \(q\) along a curve
\(\gamma: [0, \varepsilon] \to M\) is approximated by
\(P_t X_p - X_q \approx t \nabla_{\dot{\gamma}(0)}X\). Applying this
approximation to the above yields \begin{align}
    &P_{(\varepsilon,\delta)}^{(\varepsilon,0)} P_{(\varepsilon,0)}^{(0,0)} X_{(0,0)} - P_{(\varepsilon,\delta)}^{(0,\delta)} P_{(0,\delta)}^{(0,0)} X_{(0,0)}.\\\
    &\approx \varepsilon(P_{(\varepsilon,\delta)}^{(\varepsilon,0)}(\nabla_{\partial_s}X)_{(\varepsilon, 0)}
    - (\nabla_{\partial_s}X)_{(\varepsilon, \delta)})
    - \delta(P_{(\varepsilon,\delta)}^{(0, \delta)}(\nabla_{\partial_t}X)_{(0, \delta)}
    - (\nabla_{\partial_t}X)_{(\varepsilon, \delta)})\\\
    &\approx \varepsilon\delta (\nabla_{\partial_t}\nabla_{\partial_s} X - \nabla_{\partial_s}\nabla_{\partial_t} X).
\end{align}

In the above computation, we used the approximation
\(P_t X_p - X_q \approx t \nabla_{\dot{\gamma}(0)}X\), but even if we
had used the full Taylor expansion, we will find that the above is
precisely the quadratic term of the Taylor expansion of our original
expression.

\textbf{Exercise.} Compute the exact Taylor expansion of the difference
in parallel transports and verify the quadratic term is the same as our
approximation above.

Thus this difference
\(\nabla_{\partial_t}\nabla_{\partial_s} X - \nabla_{\partial_s}\nabla_{\partial_t} X\)
approximates the failure of parallel transport around small squares
parametrized by \((s,t)\) to commute and hence is a natural obstruction
to a point having a flat neighborhood. To get more obstructions, we
could choose coordinates \((x^k)\) and study the failure for parallel
transport to commute around the small coordinate square parametrized by
\((x^i, x^j)\) for any \(i,j\), which would be approximated by
\(\nabla_{\partial_{i}}\nabla_{\partial_{j}} X - \nabla_{\partial_{j}}\nabla_{\partial_{i}} X\).
This difference for all such pairs \((i,j)\) gives more obstructions to
local flattness (in fact, we will prove later these are the \emph{only}
obstructions). Next, we can encode these differences
\(\nabla_{\partial_{i}}\nabla_{\partial_{j}} X - \nabla_{\partial_{j}}\nabla_{\partial_{i}} X\)
in a coordinate-independent way by extending to a tensor field. In fact,
note these differences are already tensorial in \(X\).

\textbf{Exercise.} Verify that
\(\nabla_{\partial_{i}}\nabla_{\partial_{j}} X - \nabla_{\partial_{j}}\nabla_{\partial_{i}} X\)
is tensorial in \(X\) by computing \[
    \nabla_{\partial_{i}}\nabla_{\partial_{j}} (fX) - \nabla_{\partial_{j}}\nabla_{\partial_{i}} (fX) 
    = f(\nabla_{\partial_{i}}\nabla_{\partial_{j}} X - \nabla_{\partial_{j}}\nabla_{\partial_{i}} X)
\]

We now extend
\(\nabla_{\partial_{i}}\nabla_{\partial_{j}} X - \nabla_{\partial_{j}}\nabla_{\partial_{i}} X\)
to a tensor field \(R\) that takes \(3\) vector fields and outputs \(1\)
vector field. We know how this vector field should behave on the
coordinate frame:
\(R(\partial_{i}, \partial_{j})Z = \nabla_{\partial_{i}}\nabla_{\partial_{j}} Z - \nabla_{\partial_{j}}\nabla_{\partial_{i}} Z\)
for some vector field \(Z\). Thus we extend this to a tensor taking as
input arbitrary vector fields \(X = X^i\partial_i\) and
\(Y = Y^j\partial_j\) by defining \[
    R(X, Y)Z = X^iY^jR(\partial_i, \partial_j)Z 
    = X^iY^j\nabla_{\partial_{i}}\nabla_{\partial_{j}} Z - Y^jX^i\nabla_{\partial_{j}}\nabla_{\partial_{i}} Z
\] This is the \emph{Riemann curvature tensor}, which will encodes all
obstructions to a manifold being locally flat. In fact, we can rewrite
the Riemann curvature tensor in a form that is independent of
coordinates. Indeed, notice the terms appearing in the Riemann curvature
tensor are components of \(\nabla_X\nabla_Y Z\) and
\(\nabla_Y\nabla_X Z\) after product rule: \begin{align}
    \nabla_X\nabla_Y Z 
    &= \nabla_{X^i\partial_i}\nabla_{Y^j\partial_j} Z 
    = X^iY^j\nabla_{\partial_i}\nabla_{\partial_j} Z + X^i\partial_iY^j\nabla_{\partial_j}Z\\\
    &\text{and}\\\
    \nabla_Y\nabla_X Z 
    &= \nabla_{Y^j\partial_j}\nabla_{X^i\partial_i} Z 
    = Y^jX^i\nabla_{\partial_j}\nabla_{\partial_i} Z + Y^j\partial_jX^i\nabla_{\partial_i}Z
\end{align} Thus using the above identities, we can rewrite the
curvature tensor in a coordinate independent form as \begin{align}
        & X^iY^j\nabla_{\partial_{i}}\nabla_{\partial_{j}} Z - Y^jX^i\nabla_{\partial_{j}}\nabla_{\partial_{i}} Z\\\
        &= \nabla_X\nabla_Y Z - X^i\partial_iY^j\nabla_{\partial_j}Z 
        - \nabla_Y\nabla_X Z 
        + Y^j\partial_jX^i\nabla_{\partial_i}Z\\\
        &=  \nabla_X\nabla_Y Z - \nabla_Y\nabla_X Z - \nabla_{X^i\partial_iY^j \partial_j - Y^j\partial_jX^i \partial_i}Z\\\
        &= \nabla_X\nabla_Y Z - \nabla_Y\nabla_X Z - \nabla_{[X,Y]}Z
    \end{align} Therefore we have derived a coordinate free expression
for the Riemann curvature tensor! This expression makes for a much
better formal definition to work with.

\textbf{Def (Riemann Curvature Tensor).} The \emph{Riemann curvature
tensor} is the \((3,1)\) tensor
\(R(X,Y)Z := \nabla^2_{X,Y} Z - \nabla^2_{Y,X} Z\).

By how we derived this expression, we expect this to encode obstructions
to a manifold being locally flat and to quantify the failure for
parallel transport to commute. This is indeed the case and we can now
reverse the logic of this derivation to formally prove these properties.

\end{document}
